%
% File acl2018.tex
%
%% Based on the style files for ACL-2017, with some changes, which were, in turn,
%% Based on the style files for ACL-2015, with some improvements
%%  taken from the NAACL-2016 style
%% Based on the style files for ACL-2014, which were, in turn,
%% based on ACL-2013, ACL-2012, ACL-2011, ACL-2010, ACL-IJCNLP-2009,
%% EACL-2009, IJCNLP-2008...
%% Based on the style files for EACL 2006 by 
%%e.agirre@ehu.es or Sergi.Balari@uab.es
%% and that of ACL 08 by Joakim Nivre and Noah Smith

\documentclass[11pt,a4paper]{article}
\usepackage[hyperref]{acl2018}
\usepackage{times}
\usepackage{latexsym}

\usepackage{url}

\aclfinalcopy % Uncomment this line for the final submission
%\def\aclpaperid{***} %  Enter the acl Paper ID here

%\setlength\titlebox{5cm}
% You can expand the titlebox if you need extra space
% to show all the authors. Please do not make the titlebox
% smaller than 5cm (the original size); we will check this
% in the camera-ready version and ask you to change it back.

\newcommand\BibTeX{B{\sc ib}\TeX}

\title{Deliverable 1}

\author{Joshua Mathias \\
  University of Washington \\
  Seattle, WA \\
  {\tt emathias@uw.edu} 
  \\\And 
  Eric Lindberg \\
  U. of Washington \\
  Seattle, WA \\
  {\tt lindbe2@uw.edu} 
  \\\And 
  Person 3 \\
  UW \\
  Seattle, WA \\
  {\tt email3@uw.edu} 
  \\\And 
  Kekoa Riggin \\
  UW \\
  Seattle, WA \\
  {\tt kekoar@uw.edu} 
    }

\date{}

\begin{document}
\maketitle
\begin{abstract}
This is a report.
\end{abstract}

\section{Introduction}
Hi.

\section{System Overview}
Our system is great.

Facebook presents neural abstract text summarization and compares it to several baselines using Rouge~\cite{Rush:15}. Li suggests using a large amount of out-of-domain data for finding important content, but in-domain data for writing the summary in the correct style~\shortcite{Li:17}.

\section{Approach}
We approach perfection.

\section{Results}
We arrived.

\section{Discussion}
This is why our paper is great.

\section{Conclusion}
You should do what we did.

% ~\cite{Du:12}
% ~\cite{Lau:17}
% \textbf{Citations}: Citations within the text appear in parentheses
% itself, as Gusfield~\shortcite{Gusfield:97}.
% Using the provided \LaTeX\ style, the former is accomplished using
% {\verb|\cite|} and the latter with {\verb|\shortcite|} or {\verb|\newcite|}.

% \subsection{Equation}
% \label{ssec:eqn}

% \begin{equation}
% A=\pi r^2
% \end{equation}


%Here we give a simple criterion on your colored figures, if your paper has to be printed in black and white, then you must assure that every curves or points in your figures can be still clearly distinguished.

% Min: no longer used as of ACL 2017, following ACL exec's decision to
% remove this extra workflow that was not executed much.
% BEGIN: remove
%% \section{XML conversion and supported \LaTeX\ packages}

%% Following ACL 2014 we will also we will attempt to automatically convert 
%% your \LaTeX\ source files to publish papers in machine-readable 
%% XML with semantic markup in the ACL Anthology, in addition to the 
%% traditional PDF format.  This will allow us to create, over the next 
%% few years, a growing corpus of scientific text for our own future research, 
%% and picks up on recent initiatives on converting ACL papers from earlier 
%% years to XML. 

%% We encourage you to submit a ZIP file of your \LaTeX\ sources along
%% with the camera-ready version of your paper. We will then convert them
%% to XML automatically, using the LaTeXML tool
%% (\url{http://dlmf.nist.gov/LaTeXML}). LaTeXML has \emph{bindings} for
%% a number of \LaTeX\ packages, including the ACL 2017 stylefile. These
%% bindings allow LaTeXML to render the commands from these packages
%% correctly in XML. For best results, we encourage you to use the
%% packages that are officially supported by LaTeXML, listed at
%% \url{http://dlmf.nist.gov/LaTeXML/manual/included.bindings}
% END: remove



% include your own bib file like this:
%\bibliographystyle{acl}
%\bibliography{acl2018}
\bibliography{acl2018}
\bibliographystyle{acl_natbib}
% \appendix


\end{document}

